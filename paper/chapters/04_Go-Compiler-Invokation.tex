\chapter{Compiler Invocation}

Instead of manually generating optimized assembly for each function to be
compiled on the fly, the go compiler toolchain is invoked and receives the
previously  generated Go source code that was computed from the query languages
AST. This omits the complexity of implementing and maintaining several platform
specific machine code generator compiler backends while allowing the JIT
compiler to support all platforms supported by the go toolchain.

The method of invoking the compiler tool-chain has significant effects on the
startup performance, the robustness and the complexity of the JIT compiler.
This chapter highlights two possible approaches for invoking the compiler
tool-chain. 

\section{Including the Compiler Source Code}

The first idea of invoking the compiler tool-chain, is to include the source
code of the compiler as a library and simply start it while passing in the
generated go code. This does not require the compiler tool-chain to exist on
the target system and omits the overhead of starting the compiler process.
However this approach can not be used for the source code since the go compiler
is not stable nor accessible outside of the go compiler tool-chain
\cite[\textit{(gcToolchain).gc}]{gc_source} due to the usage of
\texttt{internal} packages \cite{go_internal_dir}.

\section{Invoking the local Go Compiler}

The remaining method is to start the locally available compiler tool-chain via
\ the \texttt{exec.Cmd} interface \cite[Overview]{go_os_exec}. This enables
requesting the operating system to invoke the compiler. Approaching the problem
with this method has the downside of requiring the compiler to exist on the
target system, the overhead of tasking the operating system with starting the
compiler, writing the generated code to a temporary file and compiling this
temporary file instead of doing all of the aforementioned inside of the JIT by
including the compiler as a library as introduced before.

\autoref{code:compiler_invocation} shows a simplified implementation of a
function invoking the go compiler. Error handling is omitted for the sake of
simplicity.

\begin{listing}[H]
    \begin{minted}{go}
func invokeCompiler(code string) {
    f, _ := os.CreateTemp(".", "jit_*.go")
    defer os.Remove(f)
    f.WriteString(code)
    c := exec.Command("go", "build", f.Name())
    c.Run()
}
    \end{minted}
    \caption{Tool-chain invocation}
    \label{code:compiler_invocation}
\end{listing}


\chapter{Introduction}

The query language is the singular interface for accessing, reading, creating
and removing data in a database. This requires the query language to provide a
high degree of performance in the sense of performing processing intensive
queries in a fast enough time for real-time responsiveness, especially for an
in memory data store with the aspiration for high performance.

Optimisations for database query languages are common, such as in the
embeddable Database \textit{SQLite} with the \textit{SQLite Query
Optimizer}\cite{sqlite_query_opt} and the \textit{Next-Generation Query
Planner}\cite{sqlite_query_opt} both supporting a variety of optimisations
after compiling SQL expressions to byte-code instead of walking the
AST\footnote{Abstract Syntax Tree: tree of syntax
nodes}\cite{sqlite_next_gen_query_plan}. \textit{PostgreSQL} is an other
example of a database optimising its SQL queries, using a JIT compiler\footnote{Just in
time compiler: compiling methods on demand while a program is running \cite[1.
Introduction]{java_jit_compilation}}\cite{postgres_jit}.

There are several optimisations applicable for any programming language and
query languages in particular \cite[3.3 Optimisations]{java_jit_region_based}.
Some were already applied to the query language \cite{parser2_opt}.
Implementing a JIT compiler can significantly improve the performance of a long
running highly processing intensive program and allows it to outperform
optimisations applied to the AST before walking it or compiling to bytecode
instructions and executing them in a dedicated virtual machine \cite[4.
Results]{java_jit_compilation}. 

The implementation of a JIT compiler introduces complexity into the codebase
for the generation of optimized assembly is inherently complex and time
consuming \cite[1. Introduction]{impl_jit} while also opening the door for
potential attacks \cite[2. Challenges Securing JavaScript JIT]{js_jit_fuzzing}
\cite[Table 1.]{js_jit_fuzzing}. Another issue is the platform dependence, a
JIT compiler has to generate native code for all platforms and operating
systems the database and therefore the query language support - increasing
complexity and maintenance \cite[Abstract]{java_jit_effectiveness}. Furthermore
a JIT compiler significantly increases the memory usage \cite[Fig.
1.]{brief_history_of_jit} of the interpreter as well as requiring a not to be
disregarded startup time made up of the time to start the JIT compiler and the
time to compile the language constructs to machine code. \cite[4.2.7 Breakdown
of compilation times]{java_jit_effectiveness}.

To evaluate whether the aforementioned positive aspects outweigh the negative
points and a JIT therefore does improve the query languages performance,
determines the scope of this paper.

\chapter{Plugin API}
\label{chapter:plugin-api}

The \texttt{plugin} package enables the loading of shared objects compiled with
the \newline \texttt{-buildmode=plugin} compiler flag and the resolution of
symbols contained in the plugin \cite[Overview]{go_plugin}. This allows the
compilation and loading of go code while running a program.

\section{Compiling Go Source Code to Go plugins}

As introduced above the compiler tool-chain accepts different build modes via
the \texttt{-buildmode} command line argument \cite{go_build_mode}. The build
mode for compiling a given source file to the go plugin format is named
\texttt{plugin} \cite{go_build_mode} \cite[Overview]{go_plugin}.

\autoref{code:compiler_invocation_plugin} contains a modified version of
\autoref{code:compiler_invocation}, adding the compiler flags for compiling the
generated source code passed via the \texttt{code} function parameter, to a go
plugin. Instead of producing an executable for the target architecture and
operating system the compiler now generates a shared object in the format the
\texttt{plugin} package requires.

\begin{listing}[H]
    \begin{minted}{go}
func invokeCompiler(code string) {
    f, _ := os.CreateTemp(".", "jit_*.go")
    defer os.Remove(f)
    f.WriteString(code)
    pre := strings.TrimSuffix(f.Name(), ".go")
    c := exec.Command(
        "go", "build", "-buildmode=plugin", "-o", pre, f.Name())
    c.Run()
}
    \end{minted}
    \caption{Tool-chain invocation with plugin compilation}
    \label{code:compiler_invocation_plugin}
\end{listing}

\section{Embedding Go plugins}

The loading of plugins and the resolution of plugins uses the API exposed by
the previously introduced \texttt{plugin} package.

\autoref{code:compiler_invocation_plugin_open} modifies
\autoref{code:compiler_invocation_plugin} for opening the previously compiled
plugin. Once opened the \mintinline{go}{*plugin.Plugin} structure can be used
for resolving exported functions and variables included in the plugin. After
resolving a symbol\footnote{a symbol refers to a function, constant or
variable} its type is \mintinline{go}{any}, therefore the function
\mintinline{go}{Main}\footnote{for the sake of this explanation the generated
code in the function parameter \texttt{code} is assumed to be contained in the
\mintinline{go}{Main} function} has to be cast to \mintinline{go}{func()}
before the go type system allows a function call. Upon type casting the
function is called and the generation, compilation and calling workflow of the
JIT compiler is concluded.

\begin{listing}[H]
    \begin{minted}{go}
func invokeCompiler(code string) {
    f, _ := os.CreateTemp(".", "jit_*.go")
    defer os.Remove(f)
    f.WriteString(code)
    pre := strings.TrimSuffix(f.Name(), ".go")
    c := exec.Command(
        "go", "build", "-buildmode=plugin", "-o", pre, f.Name())
    c.Run()

    plug, _ := plugin.Open(pre)
    // assumes generated code lives in func Main()
    symbol, _ := plug.Lookup("Main")
    Main, _ := symbol.(func())
    Main()
}
    \end{minted}
    \caption{Plugin compilation, plugin opening and function resolution}
    \label{code:compiler_invocation_plugin_open}
\end{listing}

\section{Trade-offs, Issues and Considerations}

The \texttt{plugin} package provides the program with the unique ability to
allow for high performance on the fly code compilation and execution. It
therefore fits the use case of a query language implementation well.

However the \texttt{plugin} package bears several downsides
\cite[Warnings]{go_plugin}, primarily the missing portability due to the
package only supporting Linux, FreeBSD and MacOS. Another disadvantage is the
strict requirement of both the host application and all plugins needing to be
compiled with the same tool-chain version and build-tags - this is particularly
difficult in the case of this JIT, due to the requirement of the existence of
the local compiler that will most certainly not be of the exact same version as
the compiler used for compiling the host application. Is the previously
mentioned not strictly ensured runtime errors can occur. A further drawback is
the increased difficulty of reasoning about program and plugin initialisation
for the special \mintinline{go}{func init()} function is called upon opening a
plugin \cite[Overview]{go_plugin}, possibly opening the program up to race
conditions and similar critical bugs due to global state initialisation
\cite[The init function]{go_effective}.

\chapter{Just in Time Compilation}

Just in time compilation refers to the process of determining whether a
segregated chunk of code is considered ``hot''\footnote{hot in the context of
just in time compilation refers to a code path or a segment of code that is
executed massive amount of times \cite{jvm_ibm_optlevel, jvm_efficient}} and
compiling this code segment into operating system and architecture specific
machine code ad hoc. This machine code is then loaded into the memory of the
interpreters runtime and executed instead of interpreting the code chunk
\cite{jvm_efficient}. The details of just in time compilation, meta tracing,
categorizing code segments as ``hot'', improving the performance of the just in
time compiler and error handling are explored in this chapter.

Contrary to the previously introduced definition of a just in time
compiler in the context of programming language interpreters, go does
not support dynamically loading machine code into memory and executing
these chunks. The mitigation for this is introduced and explained in
\autoref{chapter:plugin-api}.

\section{Meta-tracing \& \texttt{JIT\_CONSTANT}}
\label{sec:meta-tracing-jit-constant}

\begin{listing}[H]
    \begin{minted}{go}
// Function represents a function in the interpreter runtime
type Function[V any] struct {
    // ...

    // Counter stores the amount of calls made to the function
    Counter int
}
    \end{minted}
    \caption{\mintinline{go}{Function[V any] struct} type with meta-tracing}
    \label{code:meta-tracing-counter}
\end{listing}

Meta-tracing refers to the process of tracking the actions of the programming
language interpreter \cite[4.1 Meta-tracing]{bolz2015impact}. The interpreter
uses this functionality to determine the amount of invocations of a function
and updates the \mintinline{go}{Function.Counter} field accordingly, see
\autoref{code:meta-tracing-counter}. Once this counter reaches the threshold
defined in the \texttt{JIT\_CONSTANT} (see \autoref{code:jit-constant}) the
\mintinline{go}{type Function[V any] struct} instance is passed to the just in
time compiler compilation queue, in which it will be compiled with other
functions waiting to be compiled. Upon the Function being compiled the
interpreter executes the output of the just in time compiler for each function
invocation instead of walking the abstract syntax tree and thus is no longer
interpreting the function, but instead uses the compiled representation.

\begin{listing}[H]
    \begin{minted}{go}
// JIT_CONSTANT sets the threshold the function invocation meta tracing counter
// has to pass for the function to be considered hot and thus compilable
var JIT_CONSTANT int = 1_000
    \end{minted}
    \caption{\texttt{JIT\_CONSTANT} definition}
    \label{code:jit-constant}
\end{listing}

This constant threshold varies from compiler to compiler. The value depends on
the performance needs and the hit the runtime performance takes upon
invoking the jit compiler. Specifics are discussed in \autoref{sec:jit-constant-discussion}.

The JIT-compiler requires some information about a function before it's being
able to start the code generation step. Not only does it require the name of
the function\footnote{Unnamed/anonymous functions or closures are named by
prefixing a closure counter with \texttt{c}, the first encountered closure
will therefore be compiled as \mintinline{go}{func c0()}}, but the names of its
arguments and the types the JIT can use to compile the given function. The
necessary fields are stored in the previously introduced \mintinline{go}{type Function[V any] struct},
specifically the \mintinline{go}{type MetaData struct}
and \mintinline{go}{type MetaDataParameter struct} structures (see
\autoref{code:meta-tracing-meta-data}).

\begin{listing}[H]
    \begin{minted}{go}
type MetaDataParameter struct {
	Name string
	Type string
}
type MetaData struct {
	Parameters []MetaDataParameter
}
// Function represents a function in the interpreter runtime
type Function[V any] struct {
    // ...

    // ArgumentNames contains the list of parameter names of the function
    ArgumentNames []string
    // Name holds the name of the function
    Name string
    // MetaData holds the necessary data for the jit to compile valid functions
    MetaData *MetaData
}
    \end{minted}
    \caption{\mintinline{go}{Function[V any] struct} type with meta data}
    \label{code:meta-tracing-meta-data}
\end{listing}

\section{Connecting the JIT to the Runtime}

To stay consistent with the builder pattern the current runtime employs, the
jit consists of a struct (see \autoref{code:jit-struct}). The jit can be
enabled by calling the \texttt{FunctionGenerator.SetJit()} (see
\autoref{sec:concurrent-compilation}) function to create an instance of
\mintinline{go}{Jit[V any] struct} and attaching to it
\texttt{FunctionGenerator}. 

\begin{listing}[H]
    \begin{minted}{go}
type Jit[V any] struct {}
    \end{minted}
    \caption{\mintinline{go}{Jit[V any] struct} type representing the just in time compiler}
    \label{code:jit-struct}
\end{listing}

After parsing the input, the runtime attempts to walk the
generated abstract syntax tree, by doing so it simultaneously generates
internal representations of the corresponding nodes for interpretation.To
connect the JIT to the runtime, the previously attached \mintinline{go}{type Jit[V any] struct} reference is passed to the internal representation of a
function while generating interpretable structures (see \autoref{code:funcgen-pass-jit}), previously introduced in
\autoref{code:meta-tracing-counter} and
\autoref{code:meta-tracing-meta-data}.

\begin{listing}[H]
    \begin{minted}{go}
type FunctionGenerator[V any] struct {
    // ...
    jit             *Jit[V]
    // ...
}
    \end{minted}
    \caption{\mintinline{go}{FunctionGenerator[V any] struct} holding a reference to the just in time compiler}
    \label{code:funcgen-ref-jit}
\end{listing}

\autoref{code:funcgen-pass-jit} not only displays passing the JIT reference
to the runtimes representation of a function, but also includes the
initialisation of the previously mentioned meta data necessary for the
compilation of a function (see \autoref{code:meta-tracing-meta-data}).

\begin{listing}[H]
    \begin{minted}[breaklines]{go}
func (g *FunctionGenerator[V]) GenerateFunc(ast parser2.AST, gc GeneratorContext) (ParserFunc[V], error) {
        // ...
        closureFunc, err := g.GenerateFunc(a.Func, GeneratorContext{am: funcArgs})
        if err != nil {
            return nil, err
        }
        return func(st Stack[V], cs []V) (V, error) {
            return g.closureHandler.FromClosure(Function[V]{
                Name:          a.Name,
                ArgumentNames: args,
                JitCompiler:   g.jit,
                Counter:       0,
            }), nil
        }, nil
    //...
}
    \end{minted}
    \caption{Passing the JIT reference to \mintinline{go}{type Function[V any] struct}}
    \label{code:funcgen-pass-jit}
\end{listing}

\section{Concurrent Compilation}
\label{sec:concurrent-compilation}

Once a function is called more than the amount specified in the
\texttt{JIT\_CONSTANT}, see \autoref{sec:meta-tracing-jit-constant}, the jit
compiler attempts to compile said function. If done procedurally, this would
stall the execution of the currently compiling function by at least the amount
of time the jit takes to walk the ast of the function, generate the
corresponding go code for each tree node, invoke the go compiler on the
generated code to compile a shared object, load the go plugin and execute the
compiled function. To minimise this performance impact on the runtime imposed
by the startup of the jit compiler, as well as the compilation of functions,
the compilation is moved to a go routine.

\begin{listing}[H]
    \begin{minted}[breaklines]{go}
func (g *FunctionGenerator[V]) SetJit() *FunctionGenerator[V] {
    ctx, cancel := context.WithCancel(context.Background())
    g.jit = &Jit[V]{
        Queue:  make(chan *Function[V], 16),
        Ctx:    ctx,
        Cancel: cancel,
    }
    go func() {
        for {
        select {
            case f := <-g.jit.Queue:
                if err := g.jit.Compile(f); err != nil {
                    log.Println("[JIT] compilation failed, skipping this function", err)
                }
            case <-g.jit.Ctx.Done():
                return
            }
        }
    }()
    return g
}
    \end{minted}
    \caption{Invoking the JIT and its concurrent compilation}
    \label{code:invoking-jit}
\end{listing}

To enable concurrent compilation the jit holds a buffered channel of type
\mintinline{go}{chan *Function[V]}, this enables the non blocking compilation
of functions and therefore falls in line with the goal of minimizing the
performance impact of the compilation and the invocation itself. Furthermore
the jit makes use of the \texttt{context} package to enable the cancelation of
any compilations the jit currently performs by the runtime upon it exiting the
interpretation of the current program. These steps are implemented in the
before mentioned function \texttt{FunctionGenerator.SetJit}, see
\autoref{code:invoking-jit}. 

In \autoref{code:jit-struct} the above explained constructs necessary for the
concurrent compilation of functions are omitted,
\autoref{code:jit-struct-with-context-and-queue} therefore extends
\autoref{code:jit-struct} to include said fields.

\begin{listing}[H]
    \begin{minted}{go}
type Jit[V any] struct {
	Queue  chan *Function[V]
	Ctx    context.Context
	Cancel context.CancelFunc
}
    \end{minted}
    \caption{\mintinline{go}{Jit[V any] struct} type with concurrency constructs}
    \label{code:jit-struct-with-context-and-queue}
\end{listing}

The consumer of the queue channel the jit compiler holds, see
\autoref{code:jit-struct-with-context-and-queue}, can be seen in
\autoref{code:invoking-jit}. The producer is defined in the
\texttt{*Function[V].Eval} function. This function is invoked by the runtime
for every invocation of said function. The runtime uses this function to handle
meta tracing, such as the invocation counter and adding a function to the
compilation queue, see \autoref{code:function-eval-compile}.

\begin{listing}[H]
    \begin{minted}[breaklines]{go}
func (f *Function[V]) Eval(st Stack[V], a V) (V, error) {
    if f.wasJit && f.JitFunc != nil {
        // compiled function gets invoked here
    }

    if !f.wasJit && f.Counter >= JIT_CONSTANT && f.JitCompiler != nil && f.Ast != nil {
        f.wasJit = true
        f.JitCompiler.Queue <- f
    }

    f.Counter++
    st.Push(a)
    return f.Func(st.CreateFrame(1), nil)
}

    \end{minted}
    \caption{\texttt{Function[V].Eval} and queuing functions for compilation}
    \label{code:function-eval-compile}
\end{listing}

\section{Function Parameters and Erasing Types}

% TODO: the issue with parameter types and parameter amounts that can be 0, 1 or more


\section{Type System Clashes}

% TODO: document the struggles with asserting types of arguments + the method
% of converting generic data into its primitive underlying types and vise versa

\section{Bailing out to the Interpreter}

% TODO: errors in the jit compile stage should not stop the execution of the
% interpreter, if an error is encountered the jit simply lets the interpreter
% do its work and skips the compilation of the specific function 


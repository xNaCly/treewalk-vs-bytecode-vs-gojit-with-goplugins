\chapter{Benchmarks}

The following sections will measure the impact the jit compilation has on
selected workloads.

Benchmark results may be subject to various influences, including the workload
on the system conducting the benchmarks, insufficiently sized data sets for
comprehensive and accurate testing, as well as comparing inherently dissimilar
benchmarks. Most of the aforementioned can be mitigated by using the
\texttt{testing} package included in the go programming languages standard
library \cite{go_testing}. 

To avoid the influences of the current workload of the system the tests are
performed many times, therefore accounting for statistical outliers and
external system influences on the test, which furthermore accounts for the
possibility of choosing insufficiently sized data sets.

By benchmarking the execution of the same given input with the JIT-compiler
enabled and with the JIT-compiler disabled an inherently comparable data set is
created due to the shared purpose of evaluating the runtime performance under
differing runtime configurations. Therefore, conducting benchmarks under the
two previously mentioned configurations and comparing the results is a valid
evaluation of the two benchmarks.

\section{Arithmetics}
% TODO: a arithmetics heavy leetcode problem

\section{String operations}
%TODO: a string concatenating heavy problem

\section{Object and Array Index}
% TODO: an object and array index heavy problem, which will be the most realistic

\section{Discussing Performance Impact} 

\section{Determining the \texttt{JIT\_CONSTANT}}
\label{sec:jit-constant-discussion}

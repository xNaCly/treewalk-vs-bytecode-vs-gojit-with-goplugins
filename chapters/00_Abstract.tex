\begin{abstract}
    The goal of this paper is to evaluate whether the usage of the Go plugin
    API is feasible for just-in-time compilation of a query language intended
    for a high performance in memory data storage. This evaluation is done
    based upon the criteria of the ease of usability, performance and the
    robustness of the resulting implementation. For the sake of comparison the
    query language as well as its features are introduced. The code a JIT
    would have generated for several heavy queries is handwritten and
    benchmarked against the same expressions evaluated with the currently
    employed tree walk interpreter. The paper explores the different
    possibilities for accessing the Go compiler, working with the Go plugin
    API and highlights several benchmarks comparing the performance of the new
    JIT compiler and the previous language evaluation implementation.

    Das Ziel dieser Arbeit besteht darin zu evaluieren ob sich das Einsetzen
    der Go plugin API für die JIT Kompilation einer Abfragesprache einer
    effizienten Datenbank, die ihre Daten ausschließlich im Arbeitsspeicher
    ablegt, rentiert. Diese Abwägung basiert auf den Kriterien der
    Verwendbarkeit, Effizienz und der Stabilität der angewendeten Lösung. Um
    den Vergleich unter realistischen Bedingungen zu testen wird vorerst die
    Abfragesprache sowie einige umfangreicher werdende Beispiele vorgestellt.
    Diese Abfragen werden dann per Hand zum Go code kompiliert den der JIT
    erzeugt hätte, dieses Ergebnis wird dann gegen die derzeitige
    Implementierung die das Evaluieren mit einem Tree-walk Interpreter umsetzt,
    verglichen. Auch werden unterschiedliche Art und Weisen des Aufrufes des Go
    compiler, die Verwendung des Go plugin paketes und Tests auf die Effizienz
    des neuen JIT und des alten Tree-walk Interpreter vorgestellt.
\end{abstract}

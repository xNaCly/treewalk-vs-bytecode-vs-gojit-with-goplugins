\chapter{Just in Time Compilation}

Just in time compilation refers to the process of determining whether a
segregated chunk of code is considered ``hot''\footnote{hot in the context of
just in time compilation refers to a code path or a segment of code that is
executed massive amount of times \cite{jvm_ibm_optlevel, jvm_efficient}} and
compiling this code segment into operating system and architecture specific
machine code. This machine code is then loaded into the memory of the
interpreters runtime and executed instead of interpreting the code chunk
\cite{jvm_efficient}. The details of just in time compilation, meta tracing,
categorizing code segments as ``hot'', improving the performance of the just in
time compiler and error handling are explored in this chapter.

Contrary to the previously introduced definition of a just in time
compiler in the context of programming language interpreters, go does
not support dynamically loading machine code into memory and executing
these chunks. The mitigation for this is introduced and explained in
\autoref{chapter:plugin-api}.

\section{Meta-tracing \& \texttt{JIT\_CONSTANT}}

\begin{listing}[H]
    \begin{minted}{go}
// Function represents a function in the interpreter runtime
type Function[V any] struct {
    // ...

    // Counter stores the amount of calls made to the function
    Counter int
}
    \end{minted}
    \caption{\mintinline{go}{Function[V any] struct} type with meta-tracing}
    \label{code:meta-tracing-counter}
\end{listing}

Meta-tracing refers to the process of tracking the actions of the programming
language interpreter \cite[4.1 Meta-tracing]{bolz2015impact}. The interpreter
uses this functionality to determine the amount of invocations of a function
and updates the \mintinline{go}{Function.Counter} field accordingly, see
\autoref{code:meta-tracing-counter}. Once this counter reaches the threshold
defined in the \texttt{JIT\_CONSTANT} (see \autoref{code:jit-constant}) the
\mintinline{go}{type Function[V any] struct} instance is passed to the just in
time compiler compilation queue, in which it will be compiled with other
functions waiting to be compiled. Upon the Function being compiled the
interpreter executes the output of the just in time compiler for each function
invocation instead of walking the abstract syntax tree and thus is no longer
interpreting the function, but instead uses the compiled representation.

\begin{listing}[H]
    \begin{minted}{go}
// JIT_CONSTANT sets the threshold which the function call meta tracing has
// to pass to be compiled
var JIT_CONSTANT int = 1000
    \end{minted}
    \caption{\texttt{JIT\_CONSTANT} definition}
    \label{code:jit-constant}
\end{listing}

% TODO: cite a source for the explanation
% TODO: explain meta tracing
% TODO: explain jit constant and why 1k was choosen

\section{Function Parameters}

% TODO: the issue with parameter types and parameter amounts that can be 0, 1 or more

\section{Reducing Compiler Invocations}

% TODO: invoking the go compiler for every function to compile is slow as fuck

\section{Bailing out to the Interpreter}

% TODO: errors in the jit compile stage should not stop the execution of the
% interpreter, if an error is encountered the jit simply lets the interpreter
% do its work and skips the compilation of the specific function 
